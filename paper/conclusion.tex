\section{Conclusion}

This study introduced a comprehensive framework for evaluating tokenization strategies, emphasizing the importance of preserving linguistic integrity while maintaining computational efficiency. By focusing on metrics such as token purity, Turkish Token Percentage (TR \%), and processing efficiency, we demonstrated that tokenization strategies significantly impact downstream model performance, particularly in morphologically rich languages like Turkish. Our analysis revealed that parameter size alone is not a definitive predictor of performance. For example, \texttt{gemma-2} (27.2 billion parameters) outperformed the larger \texttt{llama-3.1} (70.6 billion parameters) in Turkish MMLU benchmarks, highlighting the critical role of tokenization alignment with linguistic structure. Conversely, general-purpose models such as \texttt{o200k-gpt4}, while excelling in downstream performance due to extensive optimizations, exhibited lower linguistic fidelity, reflecting the trade-offs inherent in task-specific model optimization. These findings emphasize that tailored tokenization strategies, which balance linguistic preservation and computational demands, are essential for achieving robust NLP performance across diverse languages. The proposed framework is not only applicable to Turkish but also adaptable to other languages and domains, providing a foundation for optimizing tokenization methods to improve multilingual NLP applications and enhance the quality of large language models. Future research will expand this framework to include task-specific evaluations and cross-linguistic comparisons to further refine tokenization strategies for diverse linguistic contexts.
