\begin{abstract}
  Tokenization is a critical preprocessing step in natural language processing (NLP), shaping how effectively large language models (LLMs) capture linguistic and semantic nuance. This paper presents a comprehensive framework for tokenization standards that prioritize morphological and semantic integrity, especially for morphologically complex and low-resource languages. Using Turkish as a case study, we evaluate eight LLM-derived tokenizers on a subset of the Massive Multitask Language Understanding (MMLU) benchmark. Our analysis goes beyond conventional efficiency metrics—such as vocabulary size, token count, and processing speed—by incorporating linguistic token percentages and semantic purity to assess how faithfully tokenizers preserve linguistic structure. The results highlight that language-specific tokenization strategies substantially improve downstream performance, even when training data is limited, and show that larger model parameters do not inherently yield better tokenization or enhanced results. These findings underscore the need to balance computational efficiency with linguistic alignment, tailoring tokenization methods to the morphological properties of each language. The proposed framework is adaptable to various linguistic contexts, guiding the development of tokenizers that improve model accuracy, robustness, and versatility. Future work will explore advanced morphological analysis, domain-specific customization, and cross-linguistic comparisons to further refine tokenization practices and advance the state of multilingual NLP.
  
  \textbf{Keywords:} Tokenization Standards, Morphological Integrity, Semantic Fidelity, Low-Resource Languages, Multilingual NLP, Morphologically Complex Languages
  \end{abstract}